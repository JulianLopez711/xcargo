% En ella, el autor presenta y señala la importancia, el origen (los antecedentes teóricos y prácticos), los objetivos, los alcances, las limitaciones, la metodología empleada, el significado que el estudio tiene en el avance del campo respectivo y su aplicación en el área investigada.

De acuerdo con la NTC1486, la introducción es un espacio dónde el autor presenta y señala la importancia, el origen (los antecedentes teóricos y prácticos), los objetivos, los alcances, las limitaciones, la metodología empleada, el significado que el estudio tiene en el avance del campo respectivo y su aplicación en el área investigada. No debe confundirse con el resumen, ni contener un recuento detallado de la teoría, el método o los resultados, como tampoco anticipar las conclusiones y recomendaciones~\cite{NTC14862008}, y se recomienda que la introducción tenga una extensión de mínimo 2 páginas y máximo de 4 páginas.\\

Para el desarrollo del documento utilizando la plantilla en \LaTeX se recomienda la guía \textit{the not so short guide to \LaTeX}~\cite{Oetiker2016} que brinda una introducción breve pero muy completa. En donde se presentan entro otras cosas los comandos y ambeintes para el trabajo con gráficas y tablas, como las que se muestran a continuación:

\begin{figure}[htbp!]
    \caption{Muestra de inclusión de un elemento gráfico}
    \centering
    \includegraphics[width=0.8\textwidth]{Images/ImagenYoda.jpg}
    \label{fig:yoda}
\end{figure}

Así la Figura~\ref{fig:yoda} muestra una imagen que se encuentra en el directorio images, mientras la tabla~\ref{tab:tablasEx}, muestra dos ejemplos de información tabular con combinación de columnas, y combinación de filas.

\begin{table}[htbp!]
    \centering
    \caption{Ejemplo de tabla con multi columnas (arriba) y multi filas(abajo)}
    \begin{tabular}{ |p{3cm}||p{3cm}|p{3cm}|p{3cm}|  }
        \hline
        \multicolumn{4}{|c|}{Country List} \\
        \hline
        Country Name or Area Name& ISO ALPHA 2 Code &ISO ALPHA 3 Code&ISO numeric Code\\
        \hline
        Afghanistan   & AF    &AFG&   004\\
        Aland Islands&   AX  & ALA   &248\\
        Albania &AL & ALB&  008\\
        Algeria    &DZ & DZA&  012\\
        American Samoa&   AS  & ASM&016\\
        Andorra& AD  & AND   &020\\
        Angola& AO  & AGO&024\\
        \hline
    \end{tabular}\\
    \vspace{0.5cm}
    \begin{tabular}{ |c|c|c|c| } 
        \hline
        col1 & col2 & col3 \\
        \hline
        \multirow{3}{4em}{Multiple row} & cell2 & cell3 \\ 
        & cell5 & cell6 \\ 
        & cell8 & cell9 \\ 
        \hline
    \end{tabular}

    \label{tab:tablasEx}
\end{table}

Adem\'as se recomienda consultar los manuales que aparecen en al sección de aprendizaje de overleaf, por ejemplo el de \href{https://www.overleaf.com/learn/latex/Tables}{tablas}.

La presente plantilla tiene en cuenta aspectos importantes de la Norma T\'{e}cnica Colombiana - NTC 1486~\cite{NTC14862008}. Las márgenes, numeración, tamaño de las fuentes y demás aspectos de formato, deben ser conservada de acuerdo con esta plantilla, la cual esta diseñada para imprimir por lado y lado en hojas tamaño carta. Se sugiere que los encabezados cambien según la sección/capítulo del documento.\\

La redacción debe ser impersonal y genérica. La numeración de las hojas sugiere que las páginas preliminares se realicen en números romanos en mayúscula y las demás en números arábigos, en forma consecutiva a partir de la introducción que comenzará con el número 1. La cubierta y la portada no se numeran pero si se cuentan como páginas.\\

Para trabajos muy extensos se recomienda publicar más de un volumen. Se debe tener en cuenta que algunas facultades tienen reglamentada la extensión máxima de las tesis  o trabajo de investigación; en caso que no sea así, se sugiere que el documento no supere 120 páginas.\\

No se debe utilizar numeración compuesta como 13A, 14B ó 17 bis, entre otros, que indican superposición de texto en el documento. Para resaltar, puede usarse letra cursiva o negrilla. Los términos de otras lenguas que aparezcan dentro del texto se escriben en cursiva.\\

El contenido de este documento esta basado principalmente en la NTC1486~\cite{NTC14862008} y la plantilla de Tesis de maestría y doctorado de la Universidad Nacional de colombia~\cite{Unal2014DocTesis}.

