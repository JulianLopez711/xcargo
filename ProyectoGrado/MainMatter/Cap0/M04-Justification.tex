% describir brevemente que just¿ifica el desarrollo del proyecto de investigación

% incluir el alcance que tendrá el poryecto a desarrollar

% Justificación - Borrador profesional
\section{Justificación}

La implementación de XCargo responde a la necesidad operativa y financiera de mejorar la trazabilidad y la conciliación de pagos en empresas de transporte de mediana escala. La solución propuesta facilita la transparencia en los procesos de pago, reduce tiempos administrativos y aporta evidencias digitales que facilitan auditorías internas y externas.

\subsection{Contribución académica}
Desde el punto de vista académico, el proyecto contribuye con un estudio de caso sobre la integración de componentes analíticos (BigQuery) con aplicaciones web operativas, y con una evaluación práctica sobre el impacto de reglas de validación y diseño de datos en la conciliación contable.

\subsection{Contribución práctica y beneficiarios}
Los principales beneficiarios son las empresas de transporte que adopten la solución (mejoras en eficiencia y reducción de errores), los gestores contables (conciliación más rápida) y los conductores (proceso de registro de pagos más transparente y ágil). Asimismo, la empresa XCargo podrá sustentar decisiones de producto y proceso con datos reales.

\subsection{Alcance del proyecto}
El prototipo desarrollará las funciones fundamentales suficientes para validar las hipótesis: autenticación por roles, registro de pagos con carga de comprobantes, almacenamiento en BigQuery y un conjunto de consultas y reportes para conciliación. No incluye, en esta etapa, la integración con pasarelas de pago ni migraciones masivas desde sistemas legados.

% Se recomienda completar con métricas esperadas (porcentaje de reducción de tiempo de conciliaci\u00f3n, tolerancia a errores, etc.) tras la fase de evaluación.